\begin{abstract}
 The abstract goes here...
\end{abstract}



%*****************************************
\chapter{Introduction}
%*****************************************
Some short information how to get started:
\begin{itemize}
	\item Download the \texttt{tuddesign} files from \url{http://exp1.fkp.physik.tu-darmstadt.de/tuddesign/}, these are required to run any of the KOM templates!
	\item Open the main file of your template document and adjust the section containing the main document information. Afterwards, it might also be a good idea to adjust the filenames to your needs.
	\item For your convenience, common \LaTeX\ examples are included with the template and can be found in the file \texttt{content.tex}.
	\item Carefully check the comments included in the \LaTeX\ sources of the template and the manual of \texttt{tuddesign} (for general problems with \texttt{tuddesign} it is a good idea to visit the forum on their website). General \LaTeX\ problems should be resolved via the web, manuals, or the corresponding usenet groups (\texttt{de.comp.text.tex}, \texttt{comp.text.tex}).
\end{itemize}


%*****************************************
\chapter{Examples}\label{ch:examples}
%*****************************************

Bib\TeX-Test: \cite{Steinmetz2005} \citeauthor{Steinmetz2005} \citep{Steinmetz2005}

\nocite{*} % invisibly cite all that is in the bib file! (not a good idea, only for demonstration purposes!)

\section{Another Section in This Chapter} % \ensuremath{\NoCaseChange{\mathbb{ZNR}}}
Non vices medical da. Se qui peano distinguer demonstrate, personas
internet in nos. Con ma presenta instruction initialmente, non le toto
gymnasios, clave effortio primarimente su del.\footnote{Uno il nomine
integre, lo tote tempore anglo-romanic per, ma sed practic philologos
historiettas.} Chapter~\ref{ch:examples} %\autoref

\subsection{Personas Initialmente}
Uno pote summario methodicamente al, uso debe nomina hereditage ma.
Iala rapide ha del, ma nos esser parlar. Maximo dictionario sed al.

\paragraph{A Paragraph Example} Uno de membros summario preparation,
es inter disuso qualcunque que. Del hodie philologos occidental al,
como publicate litteratura in web. Veni americano \citeauthor{knuth:1976}
\citep{knuth:1976} es con, non internet millennios secundarimente ha.
Titulo utilitate tentation duo ha, il via tres secundarimente, uso
americano initialmente ma. De duo deler personas initialmente. Se 
duce facite westeuropee web, \ref{tab:example} nos clave 
articulos ha.

\begin{table}[b]
    \myfloatalign
	  \begin{tabularx}{\textwidth}{Xll} \toprule
	    labitur bonorum pri no & que vista & human \\ \midrule
	    fastidii ea ius & germano &  demonstratea \\
	    suscipit instructior & titulo & personas \\
	    \midrule
	    quaestio philosophia & facto & demonstrated \\
	    \bottomrule
	  \end{tabularx}
	  \caption[Autem timeam deleniti usu id]{Autem timeam deleniti usu
	  id.}
	  \label{tab:example}
\end{table}

\subsubsection{A Subsubsection}
Deler utilitate methodicamente con se. Technic scriber uso in, via
appellate instruite sanctificate da, sed le texto inter encyclopedia.
Ha iste americas que, qui ma tempore capital.
Sia ma sine svedese americas. Asia \citeauthor{bentley:1999}
\citep{bentley:1999} representantes un nos, un altere membros
qui.\footnote{De web nostre historia angloromanic.} Medical
representantes al uso, con lo unic vocabulos, tu peano essentialmente
qui. Lo malo laborava anteriormente uso.

\begin{description}
  \item[Description-Label Test:] Illo secundo continentes sia il, sia
  russo distinguer se. Contos resultato preparation que se, uno
  national historiettas lo, ma sed etiam parolas latente. Ma unic
  quales sia. Pan in patre altere summario, le pro latino resultato.
    \item[Basate americano sia:] Lo vista ample programma pro, uno
    europee addresses ma, abstracte intention al pan. Nos duce infra
    publicava le. Es que historia encyclopedia, sed terra celos
    avantiate in. Su pro effortio appellate, o.
\end{description}
Tu uno veni americano sanctificate. Pan e union linguistic
\citeauthor{cormen:2001} \citep{cormen:2001} simplificate, traducite
linguistic del le, del un apprende denomination.

\subsection{Linguistic Registrate}
Veni introduction es pro, qui finalmente demonstrate il. E tamben
anglese programma uno. Sed le debitas demonstrate. Non russo existe o,
facite linguistic registrate se nos. Gymnasios, sanctificate sia
le, publicate \ref{fig:example} methodicamente e qui.

Lo sed apprende instruite. Que altere responder su, pan ma, signo
studio. Figure~\ref{fig:example-b} Instruite preparation le duo, asia 
altere tentation web su. Via unic facto rapide de, iste questiones 
methodicamente o uno, nos al.

\begin{figure}[bth]
        \myfloatalign
        \subfloat[Asia personas duo.]
        {\includegraphics[width=.45\linewidth]{gfx/example_1}} \quad
        \subfloat[Pan ma signo.]
        {\label{fig:example-b}%
         \includegraphics[width=.45\linewidth]{gfx/example_2}} \\
        \subfloat[Methodicamente o uno.]
        {\includegraphics[width=.45\linewidth]{gfx/example_3}} \quad
        \subfloat[Titulo debitas.]
        {\includegraphics[width=.45\linewidth]{gfx/example_4}}
        \caption[Tu duo titulo debitas latente]{Tu duo titulo debitas
        latente.}\label{fig:example}
\end{figure}


%************************************************
\chapter{Math Test Chapter}\label{ch:mathtest} % $\mathbb{ZNR}$
%************************************************
Ei choro aeterno antiopam mea, labitur bonorum pri no. His no decore
nemore graecis. In eos meis nominavi, liber soluta vim cu. Sea commune
suavitate interpretaris eu, vix eu libris efficiantur.

\section{Some Formulas}
Due to the statistical nature of ionisation energy loss, large
fluctuations can occur in the amount of energy deposited by a particle
traversing an absorber element\footnote{Examples taken from Walter
Schmidt's great gallery: \\
\url{http://home.vrweb.de/~was/mathfonts.html}}.  Continuous processes
such as multiple
scattering and energy loss play a relevant role in the longitudinal
and lateral development of electromagnetic and hadronic
showers, and in the case of sampling calorimeters the
measured resolution can be significantly affected by such fluctuations
in their active layers.  The description of ionisation fluctuations is
characterised by the significance parameter $\kappa$, which is
proportional to the ratio of mean energy loss to the maximum allowed
energy transfer in a single collision with an atomic electron:

\[
\kappa =\frac{\xi}{E_{\mathrm{max}}} \mathbb{ZNR}
\]
$E_{\mathrm{max}}$ is the maximum transferable energy in a single
collision with
an atomic electron.
\[
E_{\mathrm{max}} =\frac{2 m_{\mathrm{e}} \beta^2\gamma^2 }{1 +
2\gamma m_{\mathrm{e}}/m_{\mathrm{x}} + \left ( m_{\mathrm{e}}
/m_{\mathrm{x}}\right)^2}\ ,
\]
where $\gamma = E/m_{\mathrm{x}}$, $E$ is energy and
$m_{\mathrm{x}}$ the mass of the incident particle,
$\beta^2 = 1 - 1/\gamma^2$ and $m_{\mathrm{e}}$ is the electron mass.
$\xi$ comes from the Rutherford scattering cross section
and is defined as:
\begin{eqnarray*} \xi  = \frac{2\pi z^2 e^4 N_{\mathrm{Av}} Z \rho
\delta x}{m_{\mathrm{e}} \beta^2 c^2 A} =  153.4 \frac{z^2}{\beta^2}
\frac{Z}{A}
  \rho \delta x \quad\mathrm{keV},
\end{eqnarray*}
where

\begin{tabular}{ll}
$z$          & charge of the incident particle \\
$N_{\mathrm{Av}}$     & Avogadro's number \\
$Z$          & atomic number of the material \\
$A$          & atomic weight of the material \\
$\rho$       & density \\
$ \delta x$  & thickness of the material \\
\end{tabular}

$\kappa$ measures the contribution of the collisions with energy
transfer close to $E_{\mathrm{max}}$.  For a given absorber, $\kappa$
tends
towards large values if $\delta x$ is large and/or if $\beta$ is
small.  Likewise, $\kappa$ tends towards zero if $\delta x $ is small
and/or if $\beta$ approaches $1$.

The value of $\kappa$ distinguishes two regimes which occur in the
description of ionisation fluctuations:

\begin{enumerate}
\item A large number of collisions involving the loss of all or most
  of the incident particle energy during the traversal of an absorber.

  As the total energy transfer is composed of a multitude of small
  energy losses, we can apply the central limit theorem and describe
  the fluctuations by a Gaussian distribution.  This case is
  applicable to non-relativistic particles and is described by the
  inequality $\kappa > 10 $ (when the mean energy loss in the
  absorber is greater than the maximum energy transfer in a single
  collision).

\item Particles traversing thin counters and incident electrons under
  any conditions.

  The relevant inequalities and distributions are $ 0.01 < \kappa < 10
  $,
  Vavilov distribution, and $\kappa < 0.01 $, Landau distribution.
\end{enumerate}


\section{Various Mathematical Examples}
If $n > 2$, the identity
\[
  t[u_1,\dots,u_n] = t\bigl[t[u_1,\dots,u_{n_1}], t[u_2,\dots,u_n]
  \bigr]
\]
defines $t[u_1,\dots,u_n]$ recursively, and it can be shown that the
alternative definition
\[
  t[u_1,\dots,u_n] = t\bigl[t[u_1,u_2],\dots,t[u_{n-1},u_n]\bigr]
\]
gives the same result.



%*****************************************
\chapter{State of the Art}\label{ch:stateOfTheArt}
%*****************************************
\section{Overview}
In this chapter, the history and the state of the art of pervasive games in general and exergames in special will be presented. Popular representatives like <em>Ingress</em> [33] by <em>Google</em> or <em>Pokémon GO</em> [34], published by <em>Nintendo</em> (TODO really? Not Niantic both?) will be looked at.

Additionally, some interesting approaches about vehicle recognition and number plate region detection will be discussed as the concept presented in this thesis also includes automatic vision-based vehicle recognition.

\section{Pervasive Games}
\subsection{Definition}
Although the term pervasive gaming is becoming more and more popular within the last years, its definition is still vague [39]. One is calling them a new type of digital games that combine game reality and physical reality within the gameplay [46].

Another definition says, that it is a game, in which the game world exists beside the everyday environment and that never stops but surrounds the player 24 hours a day [47].
E. Nieuwdorp (TODO name) defined pervasive games as special type of game that emphasizes the relation between reality and game [39].

There is some consensus in those definitions, such as the importance of the relation between the reality and the game world.

But instead of bringing up another definition to compete with the other ones, this thesis compiles definitions of some works to define the boundary of pervasive games, as fitted for this thesis’ scope.

A definition that tries to comprise the bottom line of many other definitions was made by Hock (TODO name): <em>”A pervasive game is an overlay of the real world where it (the overlay) is persistently present. It takes place in physical environments and its gameplay interacts with or is in some way related to parts of the real world. It thereby blurs the boundaries between reality and virtuality. It ideally integrates smoothly into its user’s everyday life where it ideally is omnipresent which means that the user should be able to play it at any time in any place”</em> [26].


\subsection{Beyond Traditional Games}
Montola (TODO name) tried to approach the definition problem by investigating all kinds of games that have been classified as pervasive games for different and not unique reasons. He found out, that there is no single common denominator of those reasons.

But regarding the magic circle of play (see next section TODO), each of these games broke out of this circle in a distinct manner [14].

\subsubsection{The Magic Circle of Play}
Huizinga (TODO name) defines a play as <em>some activities</em> in <em>some places</em> that are considered playfully by the players to belong to the game and not to reality [45].

Traditionally, a game takes place at <em>“certain spaces, at certain times, by certain players”</em> [14]. This is referred to as the magic circle of play.

For a pervasive game, this circle does not have a rigid border but a “permeable membrane where conventional meanings, psychological artefacts and environments, and players alike can slip through” [39].

In relation to the magic circle of play, another definition for a pervasive game is <em>”a game that has one or more salient features that expand the contractual magic circle of play socially, spatially or temporally.”</em> [14]

In the following, each of the named dimensions, in which pervasive games expand the magic circle of play, are described.

\subsubsection{Spatial Expansion}
This means, that the location, where the game can be played at, is unclear or unlimited [14]. Thinking of the example from the beginning of this thesis, that game may have started at a factory site, but the actual spread of the zombies is unclear and could be unlimited.

Montola notes that there are challenges with this kind of expansion, causing people to play the game on unwanted places like hospitals or creating hazardous situations in traffic [14].

\subsubsection{Temporal Expansion}
In classic games, a play session has a well defined start and end time. Pervasive games can break these limits. Not only does the temporal expansion obfuscate the start and end time of a game, it also blurs the border between everyday life and gaming by leaving the player in the dark about if he is playing at the moment or not [14].

<em>”Boring moments of life can be enchanced by any mobile game, but temporal expansion reaches even the moments of not-playing”<em>, because <em>”any action could be a game action”</em> [14].

\subsubsection{Social Expansion}
The social expansion obfuscates the boundary of playership [14]. Thinking back to the example of <em>The Running Dead</em> in the beginning of this thesis, it wasn’t clear who is a participant and who isn’t. A noninvolved person could have watched a persecuted player and could have tried to help him. So the noninvolved person would become a participant without even knowing about it himself.

TODO: Add exergames definition illustration

\subsubsection{Subcategories of Pervasive Games}
There are many subcategories of pervasive games, such as immersive games, serious games or exergames, to name a few. Immersive games combine spatial, temporal and social expansion. In this genre, it’s hard to differentiate games from non-games, because everything can be seen as an immersive game [42].

Serious games are, simply said, more than fun [44]. They blend education and entertainment together in one game experience [40]. The term education is not limited to a specific domain but can be all kinds of added value other than pure entertainment and fun. Göbel et al. (TODO name of source 41) call it a broad spectrum of application domains ranging from training, simulation and education to sports and health [41].

The last two domains form the subcategory of exergames. Because exergames are a subject of this thesis, the next section dedicates to that genre in more detail.


\subsection{Exergames}
A subcategory of serious games then again are exergames [41] [15]. As shown in figure (TODO), exergames are serious games with healthcare exercises as added value. The term <em>exergame</em> is constructed from the words <em>exercise</em> and <em>videogame</em> [43].

The target of an exergame is to combine physical activity with fun of play. During the gameplay, the health aspect is not primarily focused but only a side-effect of playing the game. <em>“The training effect is secondary and nevertheless achieved”</em> [15]. An example for an exergame is <em>Dance Dance Revolution</em> (TODO reference). TODO: Describe Dance Dance Revolution. Bogost (TODO name) says about this game, that it produces exercise as an emergent outcome of play itself [43].

A user of the Nintendo Wii (TODO: cite Wii) wrote: “Video games can [be] and are a great way to have exercise and not even know you are burning calories” [16]. Debra Lieberman found out that there is growing evidence of exergames helping people to stay fit and to manage their weight if they frequently play them [17].

So the challenge of an exergame is to keep the player motivated to do sports. As in the example at the beginning of this thesis (<em>The Running Dead</em>), we all ran away when the mob of zombies came running towards us. We did not think of doing sports in that moment. Maybe some even did not think of playing a game. Our only thought was to escape.

\subsubsection{Mobile Exergames}
Digital games are mostly completely experienced through a fixed screen. This is an obstacle to immerse into a game [38], because the player is bound to his chair and his perception is usually only stimulated by the computer screen and a sound system. In pervasive games, the interface is no longer bound to a fixed screen and to common computer periphery like a mouse and a keyboard. The player rather interacts with real world objects and real persons [39], whereas the interface of the game isn’t well defined. Any real world object could be a game object, and any real person could be a participant [14].

There is a special type of exergames, the mobile exergames, in which the player is not bound to a desktop or to a classical game board to play the game. Mobile exergames are played with a mobile device (such as a smartphone or a portable video game console). They are usually played outside and often use the location of the player to track the progress of the exercise. In most cases, the player has to move to succeed in the game, such as in TODO examples: Zombies RUN, Pokémon GO, StreetConqAR, more!!!

These games do not take place at special locations like in a gym, a laboratory or at home, separated from the real world. They are played everywhere such as on the way to school or work, at work itself, after the lunch time, after work with your kids or whenever and wherever the player (or the game, as said above for pervasive games with temporal expansion) wants to. The aim is a transformation of the player’s everyday environment into a world in play [39].


\section{Evolution of Location-based Games}
In the previous chapters, different types of pervasive games have been presented. Many of them depend on the current location of the player, e. g. Ingress (see section TODO) or Pokémon GO (see section TODO). This kind of games uses the current location as a basic part of the gameplay. Without the exact location, a player of Ingress could not conquer a portal and a hunter on the prowl for Pokémons would not find one nor would he find a Pokéstop to retrieve useful items. Concluded, these games would be absolutely unplayable without being able to track the location of the player with a reasonable accuracy.

Using GPS, this is only possible since May 2, 2000, when the <em>Selective Availability</em> was turned off. Selective Availability was a service to prevent civilians to perceive a better GPS accuracy than a 100 meters radius, reserving better accuracy for the military only. Since it was disabled, the GPS location is accurate to a few meters for anybody [48].

Suddenly, many location based games came up. For example, the term <em>Geocaching</em> [49] was formed during that time to describe a hunt for treasures (the <em>geocaches</em>) in the real world [50]. To find these geocaches, their locations are marked on a virtual map via GPS.

A few years later, the popular 80s game <em>Pacman</em> (TODO: link to original game?) was transferred into a real-life version named <em>Pac-Manhattan</em> [51], using GPS sensors to track the position of the players.

With the launch of the iPhone in 2007, an era began where people always have a device at hand that is not only capable of tracking the GPS location but that is also powerful enough to run useful apps on it. This caused another hype of location-based applications.

One of it, heavily relying on GPS-based location, is Foursquare [52]. The service makes personalized recommendations of places to a user. These recommendations depend on the user’s current location and his profile. The service exists as a mobile app and as a website. Providing an iPhone app in 2009, Foursquare was one of the pioneers that combined a serious mobile application with a location-based service.

At that time, the app had a check-in feature to virtually check into specific locations. This game-like feature motivated users to visit different places to check-in and leave a Tip.

Another location-based game, that became very popular, is <em>Ingress</em> (see chapter TODO). Ingress also uses the smartphones of the players to track their location and let them conquer virtual entities (named portals), that have been spread over the world.

A new member of location-based games is <em>Pokémon GO</em> (see chapter TODO). Like in Ingress, players have to move in the real world, using their smartphone to discover virtual objects (like Pokémons or Pokéstops), that have been placed on real-world locations.

The real-world locations, where entities are placed in the virtual world, play an important role for these games. They have to be well-chosen to fit the purpose of the game and the respective gameplay. These special locations are called <em>anchors</em>.

\subsubsection{Anchors}
For location-based games, the real world is overlain by the virtual game world. Anchors are well defined positions, where these two worlds are blurred [26]. The word <em>anchor</em> has been introduced by Hock (TODO name) [26]. In literature, there is no fixed term for this. Other works also refer to this as <em>artifacts</em> [54] or as <em>markers</em> [53]. This thesis uses the term anchor [26], because its semantic meaning describes very well what an anchor does: It anchors the virtual world into the real world.

The placement of anchors is very important for a game and is directly associated with the user experience. Because anchors are those positions, where the gameplay usually takes place, players may easily be disappointed if an anchor is either too far away to reach it or if it is at a location not (without further ado) accessible for civilians (such as military areas, airports or the ocean). So anchors have to be placed in a way that they are accessible for all players.
Another thing is the content, that is linked to an anchor. For example, the game <em>Escape From the Tower</em> [55] lets its players reenact the escapes of some popular prisoners from the London Tower. Therefore, anchors are placed all over the tower and its surroundings with associated media. At a prison cell, this media tells the player something about the prisoner. On the staircase, the media is related to a fled, that occurred. If the content does not fit to the respective location, players will soon quit the game.

Reid (TODO name) [54] classifies the way, anchors are placed, into three categories, that will be explained in the following.

\paragraph{User-placed at Runtime}
With this method, the user places his own anchors. For this, the game has to provide some kind of editor, with which the user creates the content he needs to play the game.

The advantages of this method are, that the game can be played everywhere, because it does not depend on predefined anchors. Another advantage is, that users can create the anchors such that they fit their individual needs. In Twostone [TODO source] for example, the player plays a stone-eating caterpillar that moves on a maze-like virtual track while the player moves in the real world. To win the game, the caterpillar named “Twostone” has to eat all stones, that are spread all over the track while escaping from some monsters, that try to catch it. The game includes an editor, wherewith the user can design a track by walking along the paths, where the track should be created and placing anchors at corners. The game can thus be played anywhere. Also, highly motivated and sporty players can build their own large tracks covering the whole city, whereas beginners are able to design a small one for a short sport unit during the lunch time.

Drawbacks of this method are, that a user is not a designer and may not have the required artistic skills and knowledge to produce qualitative anchors. And much less in a small amount of time [56].

Additionally, if the player desires for a quick sporting activity, he might rather go jogging in the park than starting to build a track.

\paragraph{Seamful Design}
With this method, the placement of anchors depends on features of the physical infrastructure. In <em>Feeding Yoshi</em> [TODO cite], anchors are placed on the seams of wireless access points. In the area of a secure wireless access point, the player can find a hungry creature, a “Yoshi”, whereas at unsecure wireless access points, there are plantations to plant each Yoshi’s favorite fruit.

Usually, developers try to hide the limitations caused by infrastructural seams. With this method, these limitations are embedded into the gameplay [57].

Games using a seamful design to place anchors can be played at all locations, where the respective infrastructure exists. Therefore, the accessibility depends on the choice of the kind of the infrastructure. Nowadays, unsecure wireless access points are rather seldom, hence the gameplay of Feeding Yoshi suffers from this choice.

\paragraph{Designer-placed}
With this approach, all anchors are placed by the game designer. Due to the designer’s skills and insight to the respective game, the anchors are usually well-placed and harmonize with the gameplay.

As a drawback, this method is the most static and expensive one, consuming time and money until all needed anchors are placed. As a result, most games with designer placed anchors are tailored for a specific scenario, such as <em>Escape from the Tower</em> [55], which was mentioned above. The anchors in such games are often strongly related to their location. Adapting this game for another location would involve much effort.

Additional to those categories introduced by Reid (TODO name, not source), this thesis names two further categories.

\paragraph{Crowd-based Placement}
Crowd-based placement of anchors is akin to user placement, with the difference that content created by the crowd is made public for everyone. This way, the game thrives on a lively crowd of users, frequently creating new content.

Twostone, that has been mentioned before, is actually relying on crowd-based anchors, because after creating a new track, a player can opt to make the track public to be used by other players. But as mentioned in the beginning, this does only work if there is a crowd creating that content. During the release of a game, it might get challenging to build up a crowd despite the fact that there is no content yet.

Another problem is, that with a crowd, there is no control of the content. After some time, the game might contain anchors at hazardous locations such as airports or highways or it might contain pictures of morally questionable content, such as swastikas or of adult kind.

Ingress uses a crowd-based placement, combined with a designer to check the content for its propriety. Nevertheless, Ingress is only playable in highly populated areas while portals (the anchors) only exist in these regions (see Ingress map below TODO). That’s because users are allowed to create anchors, indeed, by taking a photo of the desired new location, marking it on a map and sending this along with a short description to the development team of Ingress. So most requests are made for regions with a lot of citizens. But it takes several weeks, until a new portal gets accepted by the designers.

\paragraph{Automatically-placed}
This scheme describes content, that is generated automatically, obeying some creation-rules. Automatic placement of anchors helps to spread them all over the world in a fraction of time and money, a designer would need.

Actually, the seamful design is a sub-scheme of automatic placement, whereas the creation-rule is to place anchors at infrastructural seams (e. g. of wireless access points).
The rule can be well-defined, such as with a procedural content generation rule, including the use of generative grammars, spatial algorithms or pseudo-random number generators [56].
The rule can also be more complex like <em>place an anchor at each gas station marked by Google Maps</em>, which already includes some locational context.

Automatic placement also includes <em>random</em> placement. But simply placing anchors at random locations will seldomly have the result of well-placed anchors semantically fitting to their content and being accessible for everyone (e. g. oceans and deserts).

Hock (TODO name) proposed an automatic placement based on real-world objects, namely street name signs [26]. In his game <em>StreetConqAR</em>, the players have to conquer streets, identified by the name on their street name sign. Because StreetConqAR is similar to the game RacecAR GO, which is presented in this thesis, there’s an extra section for this (see TODO).

The definition of an anchor says, that it is a “well-defined position”. In the following, the term anchor is not only the position but also refers to the real-world object itself to describe the kind of blurring of reality and virtuality and the strong relationship between the anchor and its position. This blurring takes place via an object at a well defined position.

Later on, this thesis will define the term of a <em>natural anchor</em> to solve this ambiguity (see TODO concept chapter).

The advantage of Hock’s (TODO name) approach is, that it not only saves the expanses of the a priori design of the anchors, because street name signs are already there. Also the spreading of these signs is very useful for a location-based exergame. There are very few populated regions with no street name signs.

TODO: Mehr? Vllt. was über seine Suche nach passenden Anchor-Objekten und deren Eigenschaften schreiben. Hat er da irgendwas kategorisiert? Oder erst in Concept auf der Suche nach eigenen Anchors?









%*****************************************
\chapter{Concept}\label{ch:concept}
%*****************************************
TODO
Reference to State of the Art in chapter \ref{ch:stateOfTheArt}.


%*****************************************
%*****************************************
%*****************************************
%*****************************************
%*****************************************